\Clear{426.79134pt}{%
\let\la\lambda
\noindent\textbf{Remark 3.}\enspace
We remark that; when the ratio $h/\la$ tends to 0, the expression $\la
L(r,s)=-(s-r)/(4\left(\frac{h}{\la}\right)^2+(r-s)^2)$ tends to
$1/(r-s)$ which is a singular function. This means that the expression
$\la L(r,s)$ is not well behaved for the small values of
$h/\la$. Consequently, for the solution to converge, the integrals of
(10) and (11) must be evaluated with a large number of
nodes. In our numerical applications (cf. section 5), we
use 100 nodes to evaluate these integrals. With the smallest value of
$h/\la=0.02$, the convergence is good with $N=20$.
}
\Clear{426.79134pt}{%
\noindent\textbf{Theorem 2.}\enspace\itshape
For system (8), consensus can be achieved with
$\|T_{\omega z}(s)\|_{\infty}<\gamma$ if there exist a symmetric
positive definite
 matrix $P\in \mathcal{R}^{(n-1)\times (n-1)}$ and a scalar $\mu>0$ satisfying
\setcounter{equation}{9}
\begin{eqnarray}\label{10}
\Gamma=\begin{bmatrix}
-\bar{L}^TP-P\bar{L}+U_1^TU_1+\mu \bar{E}&PU_1^TE_1&PU_1^T\\
E_1^TU_1P&-\mu I&0\\ U_1P&0&-\gamma^2I\end{bmatrix}<0,
\end{eqnarray}
where $\bar{L}=U_1^TLU_1$ and $\bar{E}=U_1^TE_2^TE_2U_1.$
}
\Clear{426.79134pt}{%
\noindent\textbf{Proof of Theorem 2.}\enspace Proof follows
straightforward from Lemma 3 and Theorem 1. However, it should be
emphasized that all possible $\bar{L}_{\sigma(t)}$ should share a
common Lyapunov function $V(\delta)=\delta^T(t)P\delta(t)$ (see
the proof of Lemma 3 in Appendix A). \hfill$\square$
}
\Clear{426.79134pt}{%
 \begin{enumerate}[1.]
 \item The enumerate environment starts with an optional
   argument `1.' so that the item counter will be suffixed
   by a period.
 \item If you provide  parentheses to the number, the
   output will have only one parentheses for all the item
   counters.
 \item You can use `(a)' for alphabetical counter and '(i)' for
   roman counter.
  \begin{enumerate}[a)]
    \item Another level of list with alphabetical counter.
    \item One more item before we start another.
    \begin{enumerate}[(i)]
     \item This item has roman numeral counter.
     \item Another one before we close the third level.
    \end{enumerate}
    \item Third item in second level.
  \end{enumerate}
 \item All list items conclude with this step.
\end{enumerate}
}
\Clear{284.52756pt}{%
\lmrgn=4em
 \begin{enumerate}[Step 1.]
  \item This is the first step of the example list.
\item Obviously this is the second step.
\item The final step to wind up this example.
 \end{enumerate}
}
\Clear{284.52756pt}{%
\centering
 {\ttfamily\bs includegraphics[width=3in,angle=45]\lb
 tiger.pdf\rb}\\
 \hspace*{-1cm}
 \includegraphics[width=3in,angle=45,origin=c]{tiger.pdf}\\
 \raggedright
 \textbf{Fig.~1.}~~More details on the usage of {\ttfamily\bs
 includegraphics} can be found in the \textsf{grfguide.ps} of the
 \LaTeX{} documentation.
}
\Clear{426.79134pt}{%
\raggedright
\begin{enumerate}[{[1]}]
\item Knuth, D.E., \emph{TeX: The Program}, Computers \&
Typesetting; B., 1995, Addisson-Wesley Publishing Co., Inc., New
York.
\end{enumerate}
}
