\documentclass[a4paper, zihao = -4, linespread = 1]{ctexart}

\usepackage[normalem]{ulem} %下划线包
\usepackage{lettrine} %首字下沉包
\usepackage{indentfirst} %首行缩进包
\usepackage{fontspec} %选择本地安装的字体包
\usepackage{xcolor} %字体颜色包

\renewcommand{\refname}{参考文献}
\usepackage[numbers, sort&compress, super, square]{natbib}

\usepackage[colorlinks, bookmarksopen = true, bookmarksnumbered = true]{hyperref} %标签引用包

\begin{document}

	Hello, world!
	你好,世界!\par

	\LaTeX{} Studio

	$\backslash$ \textbackslash \texttt{\char92}

	a $\sim$ b
	a\~ b
	a\~{} b
	a\textasciitilde b

	% 引号
	``\thinspace `Max' is here.''

	% 破折、省略号与短横
	daughter-in-law
	page 1--2
	Listen---I'm serious.
	\ldots %省略号

	%下划线与删除线
	\uline{下划线} \\
	\uuline{双下划线} \\
	\dashuline{虚下划线} \\
	\dotuline{点下划线} \\
	\uwave{波浪线} \\
	\sout{删除线} \\
	\xout{斜删除线}

	% 角度和温度
	$30\,^{\circ}$三角形 \\
	$37\,^{\circ}\mathrm{C}$

	% 千位分隔位
	\mbox{1\,000\,000} % 防止断行

	% 换行与分段
	% 正文中想换行,直接使用两个回车
	% 插入一个空白段落
	第一段

	\mbox{}

	第二段 \par
	第三段

	% 段落之间的距离
	\setlength{\parskip}{0pt}

	% 首字下沉
	\lettrine{T}{his} is an example. Hope you like this package, and enjoy your \LaTeX\ trip!

	% 颜色设置
	{\color{red!70} 百分之七十红色} \par
	{\color{-red} 红色的互补色}

	% 标签与引用
	\section{质能公式\texorpdfstring{$E=mc^2$}{E=mc\textasciicircum 2}}

	% 脚注
	\footnote{This is a footnote.}

	% 援引环境
	鲁智深其师有偈言曰:
	\begin{quote}
	逢夏而擒,遇腊而执。
	\end{quote}
	圆寂之后,其留颂曰:
	\begin{quotation}
	平生不修善果,只爱杀人放火。
	\end{quotation}

	This is a sample text.\cite{author1.year1,author2.year2}
	This is the text following the reference.

	\begin{thebibliography}{99} % 99表示以最多两位数来编号,用于对齐
		\addtolength{\itemsep}{-2ex} %用于更改行距
		\bibitem{author1.year1}Au1. ArtName1[J]. JN1. Y1:1--2
		\bibitem{author2.year2}Au2. ArtName2[J]. JN2. Y2:1--2
	\end{thebibliography}


























\end{document}
